\documentclass{article}
\usepackage{amsmath} % For align environment
\usepackage{amssymb} % For additional math symbols
\usepackage{amsfonts} % For math fonts
\usepackage{geometry} % To adjust page layout
\geometry{a4paper, margin=1in} % Setting page size and margins
\usepackage{graphicx} % For including graphics
\usepackage{hyperref} % For hyperlinks

 \newcommand{\squareparenthesis}[1]{\left[#1\right]} 
 \newcommand{\roundparenthesis}[1]{\left(#1\right)} 
 \newcommand{\curlyparenthesis}[1]{\left\{#1\right\}} 
 %\newcommand{\bm}{\boldmath}
 \newcommand{\PJ}[2]{$(P,J)=(#1,#2)$}
 \newcommand{\pj}[2]{$(p,j)=(#1,#2)$}
 \newcommand{\ex}{\bm{e}_x}
 \newcommand{\ey}{\bm{e}_y}
 \newcommand{\ez}{\bm{e}_z}
 \newcommand{\er}{\bm{e}_r}
 \newcommand{\RNa}{R_{Na}}
 \newcommand{\RTa}{R_{Ta}}
 \newcommand{\epar}{\bm{e}_\parallel}
 \newcommand{\eperp}{\bm{e}_\perp}
 \newcommand{\etheta}{\bm{e}_\theta}
 \newcommand{\ephi}{\bm{e}_\phi}
 \newcommand{\kperp}{k_\perp}
 \newcommand{\lperp}{\ell_\perp} 
 \newcommand{\lperpa}{l_{\perp a}}
 \newcommand{\kpar}{k_\parallel}
 \newcommand{\kvec}{\bm{k}}
 \newcommand{\rhose}{\rho_{se}}
 \newcommand{\Rdot}{\dot{\bm{R}}}
 \newcommand{\vpar}{v_\parallel}
 \newcommand{\Nvpar}{N_{v_\parallel}}
\newcommand{\vpardot}{\dot{v}_\parallel}
 \newcommand{\vperp}{v_\perp}
 \newcommand{\vperpa}{v_{\perp a}}
 \newcommand{\vpara}{v_{\parallel a}}
 \newcommand{\vtha}{v_{tha}}
 \newcommand{\taua}{\tau_a}
 \newcommand{\FaM}{F_{aM}}
 \newcommand{\Fa}{F_{a}}
 \newcommand{\fa}{f_{a}}
 \newcommand{\Cab}{C_{ab}}
 \newcommand{\phibar}{\bar \phi}
 \newcommand{\psibar}{\bar \psi}
 \newcommand{\chibar}{\bar \chi}
 \newcommand{\Upsbar}{\bar \Upsilon}
 \newcommand{\Apar}{A_\parallel}
 \newcommand{\dBpar}{\delta B_\parallel}
 \newcommand{\KFaM}{K_{\FaM}}
 \newcommand{\omegas }{\omega_*}
 \newcommand{\Omegada }{\Omega_{da}}
 \newcommand{\Omegasa }{\Omega^*_{a}}
 \newcommand{\omegagrad}{\omega_\nabla}
 \newcommand{\omegacurv}{\omega_\kappa}
 \newcommand{\qa}{q_a}
 \newcommand{\sqpi}{\sqrt{\pi}}
 \newcommand{\spara}{s_{\parallel a}}
 \newcommand{\spar}{s_{\parallel}}
 \newcommand{\wperpa}{w_{\perp a}}
 \newcommand{\wperp}{w_{\perp}}
 \newcommand{\upara}{u_{\parallel a}}
 \newcommand{\upar}{u_{\parallel}}
 \newcommand{\sperpa}{s_{\perp a}}
 \newcommand{\sperp}{s_{\perp}}
 \newcommand{\Tperpa}{T_{\perp a}}
 \newcommand{\Tperp}{T_{\perp}}
 \newcommand{\Tpara}{T_{\parallel a}}
 \newcommand{\Tpar}{T_{\parallel }}
 \newcommand{\Pperpa}{P_{\perp a}}
 \newcommand{\Ppara}{P_{\parallel a}}
 \newcommand{\Qperpa}{Q_{\perp a}}
 \newcommand{\Qpara}{Q_{\parallel a}}
  \newcommand{\qperp}{q_{\perp}}
   \newcommand{\qpar}{q_{\parallel}}
 \newcommand{\Pparpara}{P_{\parallel a}^\parallel}
 \newcommand{\Pperpperpa}{P_{\perp a}^{\perp}}
  \newcommand{\Pparpar}{P_{\parallel}^\parallel}
 \newcommand{\Pperper}{P_{\perp}^{\perp}} 
 \newcommand{\Pparper}{P_{\parallel}^{\perp}}
 \newcommand{\Pxa}{R_{x a}}
 \newcommand{\Cpar}{\hat C_{\parallel}}
 \newcommand{\CparB}{C_{\parallel}^{B}}
 \newcommand{\Cper}{\hat C_{\perp}}
 \newcommand{\dt}{\mathrm dt}
 \newcommand{\dr}{\mathrm dr}
 \newcommand{\dtheta}{\mathrm d\theta}
 \newcommand{\dx}{\mathrm dx}
 \newcommand{\dy}{\mathrm dy}
 \newcommand{\dz}{\mathrm dz}
 \newcommand{\ddt}{\partial_t}
 \newcommand{\ddr}{\partial_r}
 \newcommand{\ddtheta}{\partial_\theta}
 \newcommand{\ddx}{\partial_x}
 \newcommand{\ddy}{\partial_y}
 \newcommand{\ddz}{\partial_z}
 \newcommand{\ddvpar}{\partial_{v\parallel}}
 \newcommand{\ddspar}{\partial_{s\parallel}}
 \newcommand{\ddspara}{\partial_{s\parallel a}}
 \newcommand{\ddepsilon}{\partial_\epsilon}
 \newcommand{\dga}{\delta g_a}
 \newcommand{\dfa}{\delta f_a}
 \newcommand{\dna}{\delta n_a}
 \newcommand{\dphi}{\delta \phi}
 \newcommand{\phitilde}{\tilde{\phi}}
 \newcommand{\gatilde}{\tilde{g}_a}
 \newcommand{\hhat}{\hat{H}}
 \newcommand{\gradperp}{\nabla_\perp}
 \newcommand{\gyroavg}[1]{\langle #1 \rangle_R}
 \newcommand{\rot}{\nabla\times}
 \newcommand{\dvr}{\nabla\cdot}
 \newcommand{\grad}{\nabla}
 \newcommand{\gradpar}{\nabla_\parallel}
 \newcommand{\bvec}{\bm{b}}
 \newcommand{\bgradb}{\bvec\nabla\bvec}
 \newcommand{\Bstar}{\bm{B}^*}
 \newcommand{\bstarpar}{B^*_\parallel}
 \newcommand{\Brstar}{\bm{B}_{R^*}}
 \newcommand{\besskvomega}{J_0\roundparenthesis{\frac{k_\perp v_\perp}{\Omega_a}}}
\newcommand{\qandq}{\quad\textrm{and}\quad}
%\newcommand{\kernel}{\mathcal{K}}
\newcommand{\kernel}{\hat K}
\newcommand{\Uoa}{\bm{U}_{0a}}
\newcommand{\Uka}{\bm{U}_{\kappa a}}
\newcommand{\Ugrada}{\bm{U}_{\nabla a}}
\newcommand{\curvB}{\bm\kappa}
\newcommand{\curv}{\nabla_\kappa}
\newcommand{\cancelterm}[2]{\underset{\textrm{#2}}{\cancel{#1}}}
\newcommand{\apjnorm}[2]{\rvert\rvert#1\rvert\rvert_a^{#2}}
\newcommand{\cso}{c_{s0}}
\newcommand{\Lperp}{L_\perp}
\newcommand{\Lref}{L_{\textrm{ref}}}
\newcommand{\kNa}{\kappa_{Na}}
\newcommand{\kTa}{\kappa_{Ta}}
\newcommand{\kN}{\kappa_{N}}
\newcommand{\kT}{\kappa_{T}}
\newcommand{\Spara}{{S}_{\parallel a}}
\newcommand{\Sperpa}{{S}_{\perp a}}
\newcommand{\Lpara}{{L}_{\parallel a}}
\newcommand{\Lperpa}{{L}_{\perp a}}
\newcommand{\Sapj}{{S}_{a}^{pj}}
\newcommand{\Napj}{{N}_{a}^{pj}}
\newcommand{\fourier}[1]{\mathcal{F}\left\{#1\right\}}
\newcommand{\ifourier}[1]{\mathcal{F}^{-1}\left\{#1\right\}}
\newcommand{\ExB}{\bm E\times\bm B}
\newcommand{\order}[1]{\mathcal O\left(#1\right)}
\newcommand{\pb}[2]{\left\{#1,#2\right\}}
\newcommand{\polynomials}[1]{\mathcal{P}_{#1}}
\newcommand{\Nmax}{{N_{\max}}}
\newcommand{\intd}{\int \mathrm{d}}
\newcommand{\Matlab}{\textsc{Matlab} }
\newcommand{\salpha}{$s-\alpha\,$}
\newcommand{\gyacomo}{\textsc{Gyacomo} }

\newcommand{\figref}[1]{Figure \ref{#1}}
\newcommand{\tocomplete}[1]{{\color{brown}#1}}
\newcommand{\toremove}[1]{\\\textit{\color{cyan}--#1--}\\}
%\newcommand{\modifi}[1]{{\color{red}#1}}
\newcommand{\modifi}[1]{{#1}}
%\newcommand{\modifii}[1]{{\color{red}#1}}
\newcommand{\modifii}[1]{{#1}}
%\newcommand{\modifiii}[1]{{\color{orange}#1}}
\newcommand{\modifiii}[1]{#1}
%\newcommand{\modifiv}[1]{{\color{orange}#1}}
\newcommand{\modifiv}[1]{#1}
%\newcommand{\reviewi}[1]{{\color{red}#1}}
\newcommand{\reviewi}[1]{#1}
%\newcommand{\reviewii}[1]{{\color{green}#1}}
\newcommand{\reviewii}[1]{#1}
%\newcommand{\reviewiii}[1]{{\color{cyan}#1}}
\newcommand{\reviewiii}[1]{#1}


\begin{document}

\section{Gyro-Moment Equations}
In the projection of the gyrokinetic Boltzmann equation onto the Hermite-Laguerre basis, the gyro-averaging operator $\hat J_0$ is written as the Bessel function of the first kind, which can be expressed in terms of Laguerre polynomials as
\begin{equation}
    \hat J_0\bigl(\sqrt{\lperp \wperp}\bigr)  =  \sum_{n=0}^\infty \kernel_i^n(\lperp)\,L_n(\wperp).
\label{eq:bess_lag}
\end{equation}
Here, $\lperp = \tau\,k_\perp^2/2$ and $k_\perp^2 = g^{xx}k_x^2 \,+\,2\, g^{xy}k_xk_y \,+\, g^{yy}k_y^2$, while the functions
\begin{equation}
    \kernel_i^n(\lperp)  =  \frac{(\lperp)^n}{n!} \, e^{-\lperp}
    \label{eq:kernel}
\end{equation}
serve as ``kernels'' that separate the configuration- and velocity-space dependences.  

When one projects the local $\delta f$ GK Boltzmann equation onto the Hermite--Laguerre basis, one obtains the following set of GM equations:
\begin{subequations}
\begin{equation}
    \label{eq:moment_hierarchy_app}
    \ddt N_i^{pj}
     + \mathcal S^{pj}
     + \mathcal M_{\parallel }^{pj}
     + \mathcal M_{\perp}^{pj}
     + \mathcal D_{T}^{pj}
     + \mathcal D_{N}^{pj}
     = \mathcal C_i^{pj},
   % \tag{\ref{eq:moment_hierarchy_app}}
\end{equation}

which writes explicitly as
\begin{align}
    % &\frac{\partial N_i^{pj}}{\partial t}
    &\ddt N_i^{pj}
    % Nonlinear ExB
    +  \sum_{n=0}^{\infty}\pb{\kernel_i^n\phi}{\sum_{s=0}^{n+j}d_{njs} N_i^{ps}}\label{eq:exbdrift}\\
    % Trapping/mirror
    & +\sqrt{\tau} \left(\Cpar\aleph_i^{p\pm1,j} - C_\parallel^B \left[(j+1)\aleph_i^{p\pm1,j}-j\aleph_i^{p\pm1,j-1}\right]\right)\label{eq:mirrforce}\\
    % Landau damping
    &+\sqrt{\tau}C_\parallel^{B}\sqrt{p}\left([2j+1]n_i^{p-1,j} -[j+1]n_i^{p-1,j+1} - j n_i^{p-1,j-1}\right)\label{eq:landdamp}\\
    % Centrifugal drift
    &+ \frac{\tau}{q_i} \Cper \squareparenthesis{\sqrt{(p+1)(p+2)} n_i^{p+2,j} + (2p+1)n_i^{pj} + \sqrt{p(p-1)}n_i^{p-2,j}}\label{eq:centforce}\\
    % Perp. Magnetic gradient drift
    &+ \frac{\tau}{q_i} \Cper \squareparenthesis{(2j+1)n_i^{pj} - (j+1)n_i^{p,j+1}-jn_i^{p,j-1}}\label{eq:magperp}\\
    % Diamagn. temperature grad.
    &+R_T ik_y\left(\kernel_i^j\squareparenthesis{\frac{1}{\sqrt{2}}\delta_{p2} -\delta_{p0}}+ \squareparenthesis{(2j+1)\kernel_i^j-(j+1)\kernel_i^{j+1}-j\kernel_i^{j-1}}\delta_{p0}\right)\phi\label{eq:diatemp}\\
    % Diamagn. density grad. + coll
    &+R_N ik_y \kernel_a^j \phi\delta_{p0} = \mathcal C_{i}^{pj},
    \label{eq:diadens}
\end{align}
with $\aleph_i^{p\pm1,j}=\sqrt{p+1} n_i^{p+1,j} + \sqrt{p} n_i^{p-1,j}$ and $n_i^{pj}=N_i^{pj}+q_i/\tau \kernel_i^j\phi\delta_{p0}$, the non-adiabatic GMs.
We also introduce the parallel curvature operator $\Cpar = R_0/(J_{xyz}\hat B)\ddz$, $J_{xyz}$ being the Jacobian of the field-aligned coordinates and $\hat B$ the normalized magnetic field amplitude, the parallel magnetic curvature $C_\parallel^{B}=\Cpar\ln B$, the perpendicular curvature operator $\Cper$. 

\end{subequations}
In Eq.~\ref{eq:moment_hierarchy_app}, $\mathcal S^{pj}$ is the nonlinear  $E\times B$
drift term (arising, for example, from the second term in Eq.~\ref{eq:exbdrift}),
which in Fourier space is expressed via Poisson brackets $\pb{\cdot}{\cdot}$ 
and the convolution of Laguerre polynomials $L_nL_j=\sum_{n=0}^{n+j}d_{njs}L_s$. 
The operator $\mathcal M_{\parallel}^{pj}$ includes trapping effects 
(Eq.~\ref{eq:mirrforce}) and Landau damping (Eq.~\ref{eq:landdamp}), 
while $\mathcal M_{\perp}^{pj}$ represents magnetic centrifugal effects (Eq.~\ref{eq:centforce}) 
and perpendicular gradient terms (Eq.~\ref{eq:magperp}). 
Finally, $\mathcal C_i^{pj}$ denotes the projection of the ion--ion collision operator (see below).  
\par
When considering an adiabatic electron response, the GM equations are closed by the quasineutrality relation,
\begin{equation}
     \Bigl(1 + \frac{q_i^2}{\tau}\bigl[1-\!\!\sum_{n=0}^{\infty}\bigl(\kernel_{i}^n\bigr)^2\bigr]\Bigr)\,\phi  - \bigl\langle \phi \bigr\rangle_{y z} 
      =  q_i\,\sum_{n=0}^{\infty}\kernel_{i}^n\, N_i^{0n}.
    \label{ch5_eq:poisson_moments_adiabe}
\end{equation}

\subsection{p=0, j=0}
\[
d_T N_i^{0,0}  + \sqrt{\tau} \left( \hat{C}_{\parallel} - C^B_{\parallel} \right) n_i^{1,0}
+ \frac{\tau}{q_i} \hat{C}_{\perp} \left( \sqrt{2} n_i^{2,0} + 2 n_i^{0,0} - n_i^{0,1} \right)
+  \left( \kernel_i^0 R_N - [\kernel_i^0 + \kernel_i^1] R_T \right) i k_y  \phi = 0
\]


\section{Fluid Model Equations}

Consider a single-charge ion species ($q_i=1$). One can expand the kernel functions $\kernel_n$ [Eq.~\eqref{eq:kernel}] for small $\tau$, namely
\begin{align}
    \kernel_0  &=  1  - \lperp\,\tau 
                  + \tfrac12\,\lperp^2\,\tau^2  + \order{\tau^3},\\
    \kernel_1  &=  \lperp\,\tau 
                  - \lperp^2\,\tau^2 
                  + \order{\tau^3},\\    
    \kernel_2  &=  \tfrac12\,\lperp^2\,\tau^2  + \order{\tau^3}.
\end{align}
These expansions assume $\lperp\,\tau \lesssim 1$ (typical of the high-$T_e$ regime).  

We present here explicitly the fluid equations derived in the gyro-moment framework:

\begin{align}
\partial_t n &+ \{(1 - \ell_\perp)\phi, n\} + \{\ell_\perp \phi, T_\perp\} + 2\tau \mathcal{C}_\perp(T_\parallel - T_\perp + n) \nonumber\\
&+ (\mathcal{C}_\parallel - \mathcal{C}_\parallel^B)\sqrt{\tau}\, u_\parallel + \left[(1 - \ell_\perp)i k_y R_N - \ell_\perp i k_y R_T\right]\phi = 0,\\
\partial_t u_\parallel &+ \{(1 - \ell_\perp)\phi, u_\parallel\} + \{\ell_\perp \phi, q_\perp\} + n \mathcal{C}_\parallel\sqrt{\tau} + 4 \tau \mathcal{C}_\perp u_\parallel \nonumber\\
&+ 6\tau \mathcal{C}_\perp q_\parallel - \tau \mathcal{C}_\perp q_\perp + 2(\mathcal{C}_\parallel - \mathcal{C}_\parallel^B)\sqrt{\tau} T_\parallel - \mathcal{C}_\parallel^B\sqrt{\tau} T_\perp = 0,\\
\partial_t T_\parallel &+ \{(1 - \ell_\perp)\phi, T_\parallel\} + \{\ell_\perp \phi, P_\parallel^\perp\} + 6\tau \mathcal{C}_\perp T_\parallel + \frac{2}{3}\tau \mathcal{C}_\perp P_\parallel^\parallel - \tau \mathcal{C}_\perp P_\parallel^\perp \nonumber\\
&+ 3\sqrt{\tau}(\mathcal{C}_\parallel - \mathcal{C}_\parallel^B) q_\parallel - 2\sqrt{\tau} \mathcal{C}_\parallel^B q_\perp + 2 \mathcal{C}_\parallel\sqrt{\tau} u_\parallel + \frac{(1 - \ell_\perp)}{2}i k_y R_T \phi = 0,\\
\partial_t T_\perp &+ \{(1 - \tau \ell_\perp)\phi, T_\perp\} + \frac{1}{2}\{\ell_\perp \phi, P_\perp^\perp\} - \tau\{(1 - \ell_\perp)\phi, n\} + 4\tau \mathcal{C}_\perp T_\perp \nonumber\\
&- \tau \mathcal{C}_\perp(n - 2P_{\perp\parallel} + 2P_{\perp\perp}) + \sqrt{\tau}(\mathcal{C}_\parallel - 2\mathcal{C}_\parallel^B) q_\perp + \mathcal{C}_\parallel^B\sqrt{\tau} u_\parallel \nonumber\\
&+ \left[\ell_\perp i k_y R_N + (3\ell_\perp - 1) i k_y R_T\right]\phi = 0,\\
\partial_t q_\parallel &+ \{\phi, q_\parallel\} - 2\sqrt{\tau}(\mathcal{C}_\parallel^B - \mathcal{C}_\parallel)P_\parallel^\parallel - 3\sqrt{\tau} \mathcal{C}_\parallel^B P_\parallel^\perp + 3 \mathcal{C}_\parallel\sqrt{\tau} T_\parallel = 0,\\
\partial_t q_\perp &+ \{\phi, q_\perp\} - \{\phi, u_\parallel\} + 2\sqrt{\tau}(\mathcal{C}_\parallel - 2\mathcal{C}_\parallel^B)P_\parallel^\perp - 2\sqrt{\tau} \mathcal{C}_\parallel^B P_\perp^\perp \nonumber\\
&+ \sqrt{\tau}(\mathcal{C}_\parallel + \mathcal{C}_\parallel^B)T_\perp + 2 \mathcal{C}_\parallel^B\sqrt{\tau} T_\parallel = 0,\\
\partial_t P_\parallel^\parallel &+ \{\phi, P_\parallel^\parallel\} = 0, \\
\partial_t P_\parallel^\perp &+ \{\phi, P_\parallel^\perp\} - \{\phi, T_\parallel\} = 0, \\
\partial_t P_\perp^\perp &+ \{\phi, P_\perp^\perp\} - \frac{1}{2}\{\phi, T_\perp\} + \frac{1}{4}\{\phi, n\} = 0.
\end{align}

These equations are complemented by the quasineutrality condition

\begin{equation}
\left(1 - 2\left[\ell_\perp - \tau\ell_\perp^2\right]\right)\phi - \langle\phi\rangle_{yz} = n + \tau\ell_\perp(T_\perp - n),
\end{equation}
where $\langle \phi \rangle_{yz}$ is the flux surface average of $\phi$, namely
\begin{equation}
    \langle \phi \rangle_{yz} = \frac{1}{\int\dz J_{xyz}}\int\dz J_{xyz}\phi(k_x,k_y,z,t)\delta_{k_y0}.
\end{equation}
and
\begin{align}
\ell_\perp &= \frac{\tau}{2} k_\perp^2,\\
k_\perp^2 &= g^{xx}k_x^2 + 2g^{xy}k_x k_y + g^{yy}k_y^2,
\end{align}

and the magnetic curvature operators defined as
\begin{align}
\mathcal{C}_\parallel^{B} &= \mathcal{C}_\parallel\ln B.
\end{align}

\end{document}
